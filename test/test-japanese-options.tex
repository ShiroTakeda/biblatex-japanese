\documentclass[10pt]{jlreq}

\usepackage{luatexja-preset}

\usepackage[bookmarks,bookmarksnumbered,colorlinks,breaklinks]{hyperref}
\hypersetup{%
citecolor=[rgb]{0.6,0,0.1},%
urlcolor=[rgb]{0,0.4,0},%
linkcolor=[rgb]{0.3,0,0},%
filecolor=[rgb]{0.3,0,0}}

% biblatex の読み込み.
\usepackage[style=authoryear,yearsuffix,nameorder]{biblatex-japanese}

% その他のbiblatex の option の指定
\ExecuteBibliographyOptions{
sortcites=true,%
}

% 文献データベースの指定.
\addbibresource{test-japanese-reverse.bib}

\biblabelsep1\zw

\begin{document}

\section{説明}

\begin{itemize}
 \item これは \texttt{biblatex-japanese} パッケージを利用したサンプルのファイル
       です.
 \item 利用している引用・参考文献スタイル
       \begin{itemize}
        \item ``\texttt{numeric}'' スタイルを利用しています.
        \item 注:biblatex-japanese のオプションで読み込むスタイルを指定していな
              いため、既定値の ``\texttt{numeric}'' スタイルが利用されています.
       \end{itemize}
 \item \texttt{yearsuffix} オプションを利用しています $\rightarrow$ 参考文献部分
       の year に「年」が付く.
 \item \texttt{nameorder} オプションを利用しています $\rightarrow$ bib ファイル
       に \texttt{test-japanese-reverse.bib} を利用しています.
\end{itemize}

% 引用部分、参考文献部分は input-text.tex というファイル.

\section{引用例}

\subsection{引用例}

\begin{itemize}
 \item 文献 \verb|\autocite{hoge2000,foobar1990}| は... \\
       \ $\rightarrow$ 文献 \autocite{hoge2000,foobar1990} は..
 \item ○○が示されている\verb|\footnote{例えば,\cite{foobar1990} と \cite{foobar1995}.}| \\
       \ $\rightarrow$ ○○が示されている\footnote{例えば,\cite{foobar1990} と \cite{foobar1995}.}.
 \item 特に平安時代の文化との関わり\verb|\autocite[25]{hoge2000}|...\\
       \ $\rightarrow$ 特に平安時代の文化との関わり \autocite[25]{hoge2000}...
 \item 考察 \verb|\autocite[30-35]{hoge2000}| は興味深い.\\
       \ $\rightarrow$ 考察 \autocite[30-35]{hoge2000} は興味深い.
 \item 論文 \verb|\footcite{hoge2000}| は非常に長い.\\
       \ $\rightarrow$ 論文 \footcite{hoge2000} は非常に長い.
\end{itemize}

複数の引用(multiple citation).
\begin{itemize}
 \item \verb|\cite{hoge2000,hoge2001,foobar1990,foobar1995}|. \\
       $\rightarrow$ \cite{hoge2000,hoge2001,foobar1990,foobar1995}.
 \item \verb|\cite{krugman91:_geog,krugman91:_bila,krugman77:_essay,krugman79:jie,brezis93:_leap}| \\
       $\rightarrow$ \cite{krugman91:_geog,krugman91:_bila,krugman77:_essay,krugman79:jie,brezis93:_leap}
\end{itemize}

\begin{tabular}{lll}
\verb|\cite{foobar1990}| & $\rightarrow$ &  \cite{foobar1990}. \\
\verb|\cite[see][100]{foobar1990}| & $\rightarrow$ &  \cite[see][100]{foobar1990} \\
\verb|\cite[see][]{foobar1990}| & $\rightarrow$ &  \cite[see][]{foobar1990} \\
\verb|\cite[100]{foobar1990}| & $\rightarrow$ &  \cite[100]{foobar1990} \\
\verb|\autocite{foobar1990}| & $\rightarrow$ &  \autocite{foobar1990} \\
\verb|\autocite[see][100-120]{foobar1990}| & $\rightarrow$ &  \autocite[see][100-120]{foobar1990} \\
\verb|\textcite{foobar1990}| & $\rightarrow$ &  \textcite{foobar1990}. \\
\verb|\textcite[see][100-120]{foobar1990}| & $\rightarrow$ &  \textcite[see][100-120]{foobar1990} \\
\verb|\footcite{foobar1990}| & $\rightarrow$ &  本文\footcite{foobar1990}. \\
\end{tabular}

\begin{tabular}{lll}
\verb|\cite{hoge2001}| & $\rightarrow$ &  \cite{hoge2001}. \\
\verb|\cite[see][100]{hoge2001}| & $\rightarrow$ &  \cite[see][100]{hoge2001} \\
\verb|\cite[see][]{hoge2001}| & $\rightarrow$ &  \cite[see][]{hoge2001} \\
\verb|\cite[100]{hoge2001}| & $\rightarrow$ &  \cite[100]{hoge2001} \\
\verb|\autocite{hoge2001}| & $\rightarrow$ &  \autocite{hoge2001} \\
\verb|\autocite[see][100-120]{hoge2001}| & $\rightarrow$ &  \autocite[see][100-120]{hoge2001} \\
\verb|\textcite{hoge2001}| & $\rightarrow$ &  \textcite{hoge2001}. \\
\verb|\textcite[see][100-120]{hoge2001}| & $\rightarrow$ &  \textcite[see][100-120]{hoge2001} \\
\verb|\footcite{hoge2001}| & $\rightarrow$ &  本文\footcite{hoge2001}. \\
\end{tabular}

\vspace{2em}

\subsection{外国語文献}

\begin{itemize}
 \item \textbf{Article}: 
       \cite{brezis93:_leap},
       \cite{ishikawa94:cje},
       \cite{Biker-2007-unemployment},
       \cite{takeda12:cce},
       \cite{takeda07:jjie},
       \cite{yamazaki13:_japan},
       \cite{takeda10:irae},
       \cite{babiker05:ej},
       \cite{yamasue09:mt},
       \cite{yamasue07:mt},
       \cite{babiker00:ep},
       \cite{parry97:ree},
       \cite{takeda19:ere},
       \cite{takeda14:eeps},
       \cite{imbens19:reas},
       \cite{attwood06:_sexed_up},
       \cite{aksin20063027},
       \cite{baez/article},
       \cite{bohringer07:ecoe},
       \cite{bohringer06:ecoe},
       \cite{krugman79:jie},
 \item \textbf{Book}:
       \cite{krugman91:_geog},
       \cite{helpman91:_int},
       \cite{fujita99:_spatial},
       \cite{ryza15:_advan},
       \cite{pearl2009Causality},
       \cite{attwood09:_mains_sex},
       \cite{attwood10:_porn},
       \cite{jones84:_hb_int},
       \cite{jones85:_hb_int},
       \cite{jones97:_hb_int},
       \cite{aristotle:poetics},
       \cite{aristotle:physics},
       \cite{aristotle:anima}
 \item \textbf{Collection}:
       \cite{westfahl:frontier},
 \item \textbf{Incollection}:
       \cite{krugman91:_bila},
       \cite{lucas76:_critique},
       \cite{DeGorter2002},
       \cite{balistreri20131513},
       \cite{westfahl:space},
       \cite{Mcconnell2005},
 \item \textbf{Unpublished}:
       \cite{ishikawa03:_ghg},
       \cite{rutherford00:_gtap},
       \cite{takeda15:_lab},
       \cite{babiker99:_kyoto},
 \item \textbf{Inbook}:
       \cite{wong95:_int},
       \cite{milne-thomson68:_theor},
       \cite{kant:kpv},
       \cite{kant:ku},
       \cite{nietzsche:historie},
 \item \textbf{Inproceedings}:
       \cite{wang89:_model},
       \cite{zhang2016Deep},
 \item \textbf{Manual}:
       \cite{brooke03:_gams},
       \cite{cms},
 \item \textbf{Techreport}:
       \cite{Peri2007},
 \item \textbf{Thesis}:
       \cite{krugman77:_essay},
       \cite{loh},
 \item \textbf{Patent}:
       \cite{sorace},
 \item \textbf{Periodical}:
       \cite{jcg},
 \item \textbf{Report}:
       \cite{chiu},
 \item \textbf{Online}:
       \cite{ctan},
\end{itemize}

\subsection{日本語文献}

\begin{itemize}
 \item \textbf{Article}:
       \cite{iwamoto91jp:haito-keika},
       \cite{Hattori00},
       \cite{Hattori01},
       \cite{Hattori02},
       \cite{40018847518},
       \cite{40017004376},
       \cite{40020418914},
       \cite{hosoda2017},
       \cite{takeda2017300208},
       \cite{120005614155},
       \cite{120005678435},
 \item \textbf{Book}:
       \cite{somusho04jp:2000io-kaisetsu},
       \cite{imai71:_micr_1},
       \cite{imai72:_micr_2},
       \cite{ito85:_inte_trad},
       \cite{kuroda97jp:keo},
       \cite{miyazawa02:_io_intr},
       \cite{katayama2001},
       \cite{markusen99jp:trade_vol_1},
       \cite{barro97jp},
       \cite{nishimura90:_micr_econ},
       \cite{arimura-takeda2012},
       \cite{arimura2017jp},
       \cite{a.___2000},
       \cite{matloff__2012},
       \cite{Boswell-2012},
       \cite{Ryza2016},
       \cite{ThoughtWorksinc.08},
       \cite{chang-2013},
       \cite{kuriyama20jp},
       \cite{Mori-201206},
       \cite{Uchida-90},
       \cite{Yokomizo-2007},
       \cite{Kuldell2018:_bio},
 \item \textbf{Incollection}:
       \cite{oyama99:_mark_stru},
       \cite{isikawa02jp:_env_trade},
       \cite{takeda12:_cge},
 \item \textbf{Inbook}:
       \cite{kiyono93:_regu_comp_1},
 \item \textbf{Online}:
       \cite{takeda13:jecon},
 \item \textbf{Unpublished}:
       \cite{naikakufu_2011},
       \cite{takeda16jp:_gtap},
       \cite{takeda07jp:cge},
 \item \textbf{PhdThesis}:
       \cite{uzawa-phd-62},
\end{itemize}

% 以下は参考文献の表示

% 全部まとめて
\printbibliography[heading=bibintoc]

% タイプ別
\printbibliography[title={Articleタイプ},type=article]
\printbibliography[title={Bookタイプ},type=book]
\printbibliography[title={Incollectionタイプ},type=incollection]
\printbibliography[title={Inproceedingsタイプ},type=inproceedings]
\printbibliography[title={Inbookタイプ},type=inbook]
\printbibliography[title={Unpublishedタイプ},type=unpublished]
\printbibliography[title={その他のタイプ},nottype=article,nottype=book,nottype=incollection,nottype=inproceedings,nottype=inbook,nottype=unpublished]
        


\end{document}
